\documentclass[10pt,a4paper]{article}
\usepackage[utf8x]{inputenc}
\usepackage{ucs}
\usepackage{amsmath}
\usepackage{amsfonts}
\usepackage{amssymb}
\usepackage{graphicx}
\usepackage{moreverb} 
\usepackage{hyperref}
\usepackage{colortbl}
\pagestyle{headings}

\begin{document}
\renewcommand{\contentsname}{Indice} 
\renewcommand\listfigurename{Lista de Figuras}
\renewcommand\listtablename{Lista de Tablas}
\newcommand\bibname{Bibliografía}
\renewcommand{\refname}{Bibliografía}
\renewcommand\indexname{Indice alfabético}
\renewcommand\figurename{Figura}
\renewcommand\tablename{Tabla}
\renewcommand\partname{Parte}
\newcommand\chaptername{Capítulo}
\renewcommand\appendixname{Apéndice}
\renewcommand\abstractname{Resumen}

\title{Protocolo de comunicación}
\author{Milton Inostroza Aguilera}
\date{19 Mayo de 2008}
\clearpage
\maketitle

\begin{abstract}

Protocolo de comunicación utilizado para envíar datos desde pyTOD hacia TOD.

\end{abstract}
\newpage
\listoftables
\newpage

\section{Protocolo de comunicación - Draft}

Se muestran las tablas bases para el registro de los objetos y para el envío de eventos:

\subsection{Identificadores bases}

La siguiente tabla muestra que cada suceso tiene un identificador en el sistema de capturación de huella.
\begin{table}[!h]
\begin{center}
\begin{tabular}{|l | c |}
\hline
Suceso & Identificador\\
\hline
Registro & 0\\
\hline
Llamada & 1\\
\hline
Asignación & 2\\
\hline
Retorno & 3\\
\hline
Instanciación & 4\\
\hline
\end{tabular}
\caption{Identificadores de sucesos} 
\end{center}
\end{table}

La siguiente tabla muestra que cada objeto tiene un identificador en el sistema de capturación de huella.
\begin{table}[!h]
\begin{center}
\begin{tabular}{|l | c |}
\hline
Id Objeto & Identificador\\
\hline
Clase & 0\\
\hline
método & 1\\
\hline
Atributo & 2\\
\hline
Funcion & 3\\
\hline
Variable local & 4\\
\hline
Probe & 5\\
\hline
Thread & 6\\
\hline
\end{tabular}
\caption{Identificadores de objetos} 
\end{center}
\end{table}
\\
\pagebreak

La siguiente tabla muestra que cada tipo de datos un identificador en el sistema de capturación de huella.
\begin{table}[!h]
\begin{center}
\begin{tabular}{|l | c |}
\hline
Type & Identificador\\
\hline
int & 0\\
\hline
str & 1\\
\hline
float & 2\\
\hline
long & 3\\
\hline
bool & 4\\
\hline
list & 5\\
\hline
tuple & 6\\
\hline
dict & 7\\
\hline
other & 8\\
\hline
\end{tabular}
\caption{Identificadores de tipo de datos} 
\end{center}
\end{table}

\subsection{Registro de objetos}

A continuación se muestra el formato que tienen el registro de los diferentes objetos dentro del capturador de huellas:\\

Se describe el registro del objeto función:\\

\begin{table}[!h]
\begin{center}
\begin{tabular}{| c | c | c | c | c | c | c |}
\hline
eventId & objectId & Id & name & argsN & \{argName_{i} & argId_{i}\}\\
\hline
int & int & int & str & int & str & int \\
\hline
\end{tabular}
\caption{Registro del objeto función} 
\end{center}
\end{table}


Se describe el registro del objeto variable local:\\

\begin{table}[!h]
\begin{center}
\begin{tabular}{| c | c | c | c | c |}
\hline
eventId & objectId & Id & parentId & name\\
\hline
int & int & int & int & str\\
\hline
\end{tabular}
\caption{Registro del objeto variable local} 
\end{center}
\end{table}


Se describe el registro del objeto clase:\\

\begin{table}[!h]
\begin{center}
\begin{tabular}{| c | c | c | c | c |}
\hline
eventId & objectId & Id & name & classBases\\
\hline
int & int & int & str & --\footnotemark[1]\\
\hline
\end{tabular}
\caption{Registro del objeto clase} 
\end{center}
\end{table}


\footnotetext[1]{Aún no se toma decisión para poder registrar las super clases que pueda tener la clase registrada.}
\newpage
Se describe el registro del objeto método:\\

\begin{table}[!h]
\begin{center}
\begin{tabular}{| c | c | c | c | c | c | c | c |}
\hline
eventId & objectId & Id & classId & name & argsN & \{argName_{i} & argId_{i}\}\\
\hline
int & int & int & int & str & int & str & int \\
\hline
\end{tabular}
\caption{Registro del objeto método} 
\end{center}
\end{table}


Se describe el registro del objeto atributo:\\

\begin{table}[!h]
\begin{center}
\begin{tabular}{| c | c | c | c | c |}
\hline
eventId & objectId & Id & parentId & name\\
\hline
int & int & int & int & str\\
\hline
\end{tabular}
\caption{Registro del objeto atributo} 
\end{center}
\end{table}


Se describe el registro del objeto thread: \\

\begin{table}[!h]
\begin{center}
\begin{tabular}{| c | c | c | c |}
\hline
eventId & objectId & Id & sysId\\
\hline
int & int & int & int\\
\hline
\end{tabular}
\caption{Registro del objeto thread} 
\end{center}
\end{table}


Se describe el registro del objeto probe: \\

\begin{table}[!h]
\begin{center}
\begin{tabular}{| c | c | c | c | c |}
\hline
eventId & objectId & Id & parentId & currentLasti \\
\hline
int & int & int & int & int\\
\hline
\end{tabular}
\caption{Registro del objeto probe} 
\end{center}
\end{table}

\newpage
\subsection{Llamada de objetos}

A continuación se muestra el formato que tienen las llamadas de los objetos función y método dentro del capturador de huellas:\\

Se describe la llamada al objeto función:\\

\begin{table}[!h]
\begin{center}
\begin{tabular}{| c | c | c | c | c | c |}
\hline
eventId & objectId & Id & argsN & typeId_{i} & argValue_{i}\\
\hline
int & int & int & int & int & value or valueId\footnotemark[1]\\
\hline
\end{tabular}
\caption{Llamada al objeto función} 
\end{center}
\end{table}

Se describe la llamada al objeto método:\\

\begin{table}[!h]
\begin{center}
\begin{tabular}{| c | c | c | c | c | c | c |}
\hline
eventId & objectId & Id & classId & argsN & typeId_{i} & argValue_{i}\\
\hline
int & int & int & int & int & int & value or valueId\footnotemark[1]\\
\hline
\end{tabular}
\caption{Llamada al objeto método} 
\end{center}
\end{table}

Es importante señalar que todas estas llamadas estan acompañadas de los siguientes datos que se describen a continuación:\\

\begin{table}[!h]
\begin{center}
\begin{tabular}{| c | c | c | c | c |}
\hline
probeId & parentTimeStampFrame & depth & currentTimeStamp & threadId\\
\hline
int & double & int & double & int \\
\hline
\end{tabular}
\caption{Coordenadas} 
\label{Coordenadas}
\end{center}
\end{table}


\footnotetext[1]{Aún no se implementa la modificación de la máquina virtual de Python para obtener el identificador único del objeto}
\newpage

\subsection{Asignación - Modificación de objetos}
A continuación se muestra el formato que tienen las asignaciones | modificaciones de los objetos variable local y atributo dentro del capturador de huellas:\\

Se describe la asignación - modificacion al objeto variable local:\\

\begin{table}[!h]
\begin{center}
\begin{tabular}{| c | c | c | c | c | c |}
\hline
eventId & objectId & Id & parentId & typeId & value\\
\hline
int & int & int & int & int & value or valueId\footnotemark[1]\\
\hline
\end{tabular}
\caption{Registro del objeto variable local} 
\end{center}
\end{table}

Se describe la asignación - modificación al objeto atributo:\\

\begin{table}[!h]
\begin{center}
\begin{tabular}{| c | c | c | c | c | c |}
\hline
eventId & objectId & Id & parentId & typeId & value\\
\hline
int & int & int & int & int & value or valueId\footnotemark[1]\\
\hline
\end{tabular}
\caption{Registro del objeto atributo} 
\end{center}
\end{table}

Es importante señalar que todas estas asignaciones - modificaciones están acompañadas de los siguientes datos \ref{Coordenadas}.

\subsection{Return}

A continuación se muestra el formato que tiene el return dentro del capturador de huellas:\\

Se describe return:\\

\begin{table}[!h]
\begin{center}
\begin{tabular}{| c | c | c | c | c | c |}
\hline
eventId & behaviorId & typeId & value & probeId & hasThrown \\
\hline
int & int & int & value or valueId\footnotemark[1] & int & bool\\
\hline
\end{tabular}
\caption{Registro de return} 
\end{center}
\end{table}

Es importante señalar return está acompañado de los siguientes datos \ref{Coordenadas}.

\footnotetext[1]{Aún no se implementa la modificación de la máquina virtual de Python para obtener el identificador único del objeto}
\end{document}
